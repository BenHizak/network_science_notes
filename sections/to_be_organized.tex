\documentclass[crop=false]{standalone}
\usepackage{graphicx} % Required for inserting images
\graphicspath{{images/}} 

\usepackage[T1]{fontenc}
\usepackage{lmodern} %sets the font
\usepackage{amsmath}
\usepackage{xcolor}
\pagecolor[rgb]{0,0,0} %black
\color[rgb]{0.9,0.9,0.9} %grey
\usepackage{hyperref}
\hypersetup{
    colorlinks=true,
    linkcolor=blue,
    filecolor=magenta,      
    urlcolor=cyan,
    pdftitle={Overleaf Example},
    pdfpagemode=FullScreen,
    }
\urlstyle{same}
\title{Network Science Notes}
\author{Ben Hizak}
\date{January 2025}
\setlength{\parindent}{0pt}
\begin{document}
\maketitle 



\subsection{Graph Elements / Notation}

Terms Network Science vs. Graph Theory


\section{Power Law Networks}
This means that the \textbf{maximum degree in a power-law network increases as a power-law of the network size n. If $\alpha=3$ the maximum degree increases with the square-root of n.} 
$k_{max} = k_{min} \, n^{1/(\alpha-1)}$
\section{Resources}
\href{https://gatech.instructure.com/courses/433540/modules}{Course Modules}
\href{https://cazabetremy.fr/Teaching/CN2021/CheatSheet/CN_CS_introduction.pdf}{Cheatsheet}
\href{https://www.albany.edu/~ravi/pdfs/sol_hw2.pdf}{CSI 445/660 – Network Science – Fall 2015}


\end{document}
